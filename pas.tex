\documentclass{article}
\usepackage[utf8]{inputenc}
\usepackage{xurl}
\usepackage[czech]{babel}

\setlength{\parskip}{\baselineskip}
\setlength{\parindent}{0pt}

\newcommand{\fakesection}[1]{%
  \par\refstepcounter{section}% Increase section counter
  \sectionmark{#1}% Add section mark (header)
  \addcontentsline{toc}{section}{\protect\numberline{\thesection}#1}% Add section to ToC
  % Add more content here, if needed.
}

\title{Pravděpodobnost a statistika}
\date{15. květen 2022}
\author{Filip Peterek}

\begin{document}

\maketitle

\section{Pravděpodobnost}

\subsection{Zakladni vzorce}

Variace bez opakovani: \[V(n,k) = \frac{n!}{(n-k)!}\]

Kombinace bez opakovani: \[C(n,k) = \frac{n!}{k!(n-k)!}\]

Permutace: \[P(n) = n!\]

Variace s opakovanim: \[V*(n,k) = n^k\]

Kombinace s opakovanim: \[C*(n,k) = C*(n+k-1, k) = \frac{(n+k-1)!}{(n-1)! * k!}\]

Permutace s opakovanim: \[P*(n_1, n_2, ..., n_k) = \frac{P(n)}{P(n_1) * P(n_2) * ... * P(n_k)} = \frac{n!}{n_1! * n_2! * ... * n_3!} \]

Prunik jevu: \[P(A \cap B) = P(A|B) * P(B)\]

Prunik nezavislych jevu: \[P(A \cap B) = P(A) * P(B)\]

Podminena pravdepodobnost: \[P(A|B) = \frac{P(A \cap B)}{P(B)}\]

\subsection {Bayesuv vzorec}

Nastal jev A, hledam pravdepodobnost, ktery z jevu $B_i$ jev A zpusobil.

\[ P(B_k|A) = \frac{P(A|B_k) * P(B_k)}{ \sum_{i=1}^{n} P(A|B_i) * P(B_i) } \]
 
\subsection{Nahodna velicina}

Stredni hodnota: 
\[ \mu = \sum_{(i)} x_i * P(x_i) \]
\[ \mu = \int_{-\infty}^{\infty} x_i * P(x_i) \]

\[ E(aX + b) = aE(X) + b \]
\[ E(\sum_i^n X_i) = \sum_i^n E(X_i) \]

Centralni moment r-teho radu:

\[ \mu_r' = \sum_{(i)} (x_i - E(X))^r * P(x_i) \]
\[ \mu_r' = \int_{-\infty}^{\infty} (x_i - E(X))^r * P(x_i) \]

Variance:

\[ D(X) = \sum_{(i)} (x_i - E(X))^2 * P(x_i) \]

\[ D(X) = \int_{-\infty}^{\infty} (x_i - E(X))^2 * P(x_i) \]

\[ D(X) = E(X^2) - (E(X))^2 \]

\[ D(aX + b) = a^2D(X) \]

Smerodatna odchylka:

\[ \sigma = \sqrt{D(X)} \]

Sikmost:

\[ \alpha_3 = \frac {\mu_3} {\sigma^3} \]

\end{document}

