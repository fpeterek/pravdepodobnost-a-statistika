\documentclass{article}
\usepackage[utf8]{inputenc}
\usepackage{xurl}
\usepackage{amsmath}
\usepackage{amssymb}
\usepackage{bm}
\usepackage[czech]{babel}

\setlength{\parskip}{\baselineskip}
\setlength{\parindent}{0pt}

\newcommand{\fakesection}[1]{%
  \par\refstepcounter{section}% Increase section counter
  \sectionmark{#1}% Add section mark (header)
  \addcontentsline{toc}{section}{\protect\numberline{\thesection}#1}% Add section to ToC
  % Add more content here, if needed.
}

\title{Pravděpodobnost a statistika}
\date{15. květen 2022}
\author{Filip Peterek}

\begin{document}

\maketitle

\section{Pravděpodobnost}

\subsection{Zakladni vzorce}

Variace bez opakovani: \[V(n,k) = \frac{n!}{(n-k)!}\]

Kombinace bez opakovani: \[C(n,k) = \frac{n!}{k!(n-k)!}\]

Permutace: \[P(n) = n!\]

Variace s opakovanim: \[V*(n,k) = n^k\]

Kombinace s opakovanim: \[C*(n,k) = C*(n+k-1, k) = \frac{(n+k-1)!}{(n-1)! * k!}\]

Permutace s opakovanim: \[P*(n_1, n_2, ..., n_k) = \frac{P(n)}{P(n_1) * P(n_2) * ... * P(n_k)} = \frac{n!}{n_1! * n_2! * ... * n_3!} \]

Prunik jevu: \[P(A \cap B) = P(A|B) * P(B)\]

Prunik nezavislych jevu: \[P(A \cap B) = P(A) * P(B)\]

Podminena pravdepodobnost: \[P(A|B) = \frac{P(A \cap B)}{P(B)}\]

\subsection {Bayesuv vzorec}

Nastal jev A, hledam pravdepodobnost, ktery z jevu $B_i$ jev A zpusobil.

\[ P(B_k|A) = \frac{P(A|B_k) * P(B_k)}{ \sum_{i=1}^{n} P(A|B_i) * P(B_i) } \]
 
\subsection{Nahodna velicina}

Stredni hodnota: 
\[ \mu = \sum_{(i)} x_i * P(x_i) \]
\[ \mu = \int_{-\infty}^{\infty} x_i * P(x_i) \]

\[ E(aX + b) = aE(X) + b \]
\[ E(\sum_i^n X_i) = \sum_i^n E(X_i) \]

Centralni moment r-teho radu:

\[ \mu_r' = \sum_{(i)} (x_i - E(X))^r * P(x_i) \]
\[ \mu_r' = \int_{-\infty}^{\infty} (x_i - E(X))^r * P(x_i) \]

Variance:

\[ D(X) = \sum_{(i)} (x_i - E(X))^2 * P(x_i) \]

\[ D(X) = \int_{-\infty}^{\infty} (x_i - E(X))^2 * P(x_i) \]

\[ D(X) = E(X^2) - (E(X))^2 \]

\[ D(aX + b) = a^2D(X) \]

Smerodatna odchylka:

\[ \sigma = \sqrt{D(X)} \]

Sikmost:

\[ \alpha_3 = \frac {\mu_3} {\sigma^3} \]

Spicatost:

\[ \alpha_4 = \frac {\mu_4} {\sigma^4} \]

Modus: nejpocetnejsi prvek, prvek s nejvyssi pravdepodobnosti

\subsection{Nahodny vektor}

Vektor, jehoz slozky jsou nahodne veliciny

Vztahy jsou ekvivalentni nahodne velicine, ale upravene pro vektor

Ukazka:

Necht $\boldsymbol{X} = (X, Y)$ je nahodny vektor. Potom plati:

\[ E(\boldsymbol{X}) = (E(X), E(Y)) \]

\subsection{Nezavislost nahodnych velicin}

Necht $\boldsymbol{X} = (X, Y)$ je nahodny vektor. X, Y jsou nezavisle, prave kdyz plati:

\[ F(x,y) = F_X(x) * F_Y(y) \]

\subsection{Kovariance a koeficient korelace}

\textbf{Kovariance $cov(X,Y)$}

\[ cov(X,Y) = E[(X - E(X)) * (Y - E(Y))] \]

Kladna hodnota kovariance: zvysi se X $\implies$ pravdepodobne se zvysi Y
Zaporna hodnota kovariance: zvysi se X $\implies$ pravdepodobne se snizi Y

\[ cov(X, Y) = E(XY) - E(X)*E(Y) \]
\[ cov(X,X) = D(X) \]
\[ cov(a_1X + b_1, a_2X + b_2) = a_1a_2cov(X,Y) \]

Jsou-li X, Y nezavisle $\implies$ $cov(X,Y) = 0$

\textbf{Korelacni koeficient $\rho(X, Y)$}

\begin{equation}
    \rho(X,Y)=\begin{cases}
    \frac { cov(X,Y) } { \sqrt{D(X) * D(Y)} }, & D(X), D(Y) \neq 0\\
    0, & \text{jinak}.
  \end{cases}
\end{equation}

Korelacni koeficient je mirou linearni zavislosti dvou slozek nahodneho vektoru.

\[ \rho(X, Y) = \rho(Y, X) \]

\[ \rho(X, X) = 1 \]

\[ X, Y \text{ jsou nezavisle } \implies \rho(X, Y) = 0 \]

Implikace, naopak predchozi vztah neplati

\[ \rho(X, Y) = 0 \implies X, Y \text{jsou nekorelovane} \]

\subsection{Alternativni rozdeleni}

Pouze dve moznosti, kazde ma svou pravdepodobnost

\[ P(X=1) = p \]
\[ P(X=0) = 1-p \]

\[ E(X) = p, D(X) = p * (1 - p) \]

\subsection{Binomicke rozdeleni}

\[ X \rightarrow Bi(n, p) \]

$n$ $\rightarrow$ velikost vyberu

p $\rightarrow$ pravdepodobnost uspechu

Provadim nezavisle pokusy (Bernoulliho pokusy), pravdepodobnost uspechu je konstantni

Binomicke rozdeleni - pravdepodobnost, ze v $x$ pokusech se objevi $y$ uspechu

Negativne binomicke rozdeleni -- pocet pokusu do $k$-teho uspechu vcetne

\[ X \rightarrow NB(k, p) \]

$k$ $\rightarrow$ k - pocet uspechu

p $\rightarrow$ pravdepodobnost uspechu


\begin{verbatim}

20 pokusu
Pravdepodobnost jednoho uspechu je 0.3
Pravdepodobnost, ze uspechu bude pet a mene ziskame

pbinom(5, 20, 0.3)

Pravdepodobnost, ze uspechu bude nad pet

1 - pbinom(5, 20, 0.3)


Pocet uspechu je 6
Pozadovana pravdepodobnost pro 6 uspechu je 0.7
Pravdepodobnost uspechu pri jednom pokusu je 0.3

qnbinom(0.7, 6, 0.3) + 6

Je treba pricist 6, bo R pocita jen neuspechy, 
kdezto my chceme vsechny pokusy

nbinom - negativne binomicke rozdeleni

\end{verbatim}

\subsection{Hypergeometricke rozdeleni}

Popisuje pocet uspechu v \textbf{zavislych pokusech}.

\[ X \rightarrow H(N, M, n) \]

N -- velikost DS

M -- pocet prvku s danou vlastnosti

n -- velikost vyberu

\begin{verbatim}

[r|d|p|q]hyper()

\end{verbatim}

Je-li $\frac{n}{N} < 0.05$, lze hypergeom. rozdeleni nahradit binomickym
s param. $n$ a $(M/N)$

\subsection{Poissonovo rozdeleni}

Modelujeme vyskyty udalosti na intervalu (plocha, cas, jakykoliv jiny interval)

\begin{itemize}
    \item \textbf{Ordinarita} -- pravdepodobnost vyskytu v limitne kratkem intevralu ($t \rightarrow 0$) je nulova
    \item \textbf{Stacionarita} -- pravdepodobnost vyskytu zavisi pouze na delce intervalu
    \item \textbf{Nezavisle prirustky} -- pocty udalosti v disjunktnich intervalech jsou nezavisle
    \item \textbf{Beznaslednost} -- pravdepodobnost vyskytu nezavisi na case, ktery uplynul od minule udalosti
\end{itemize}

\textbf{Rychlost vyskytu udalosti: $\lambda$}

\[ x \rightarrow Po(\lambda t) \]

\[ E(X) = D(X) = \lambda t \]

\[ (n > 30 \land p < 0.1) \implies Bi(n,p) \sim Po(np) \]

Priklad:

\text{Pocet vyskytu: 30 za hodinu}

\text{Sledovane obdobi: 20 minut}

\[ \lambda = 30 \]

\[ t = 20 min \]

\[ \lambda t = \frac{30}{3} = 10 \]

\begin{verbatim}

Pocet udalosti: 5
Casove obdobi: 15

ppois(5, lambda * 15)

\end{verbatim}

\subsection{Rovnomerne rozdeleni}

Pravdepodobnost je konstantni na intervalu $(a;b)$, jinde je nulova

\[ X \rightarrow R(a;b) \]

\[ E(X) = \frac{a+b}{2}, D(X) = \frac{(a-b)^2}{12} \]


\subsection{Exponencialni rozdeleni}

Mejme Poissonuv proces.

Potom \textbf{doba do vyskytu prvni udalosti, pripadne doba mezi udalostmi,} je modelovatelna exponencialnim
rozdelenim.

Bezpametove rozdeleni $\rightarrow$ doba do vyskytu udalosti nezavisi na predchozich vyskytech.

\[ X \rightarrow Exp(\lambda) \]


\begin{equation}
    f(t)=\begin{cases}
        \lambda * e^{-\lambda t}, & t > 0\\
        0, & \text{jinak}.
  \end{cases}
\end{equation}

\begin{equation}
    f(t)=\begin{cases}
        1 -  e^{-\lambda t}, & t > 0\\
        0, & \text{jinak}.
  \end{cases}
\end{equation}

\[ E(X) = \frac{1}{\lambda}, D(X) = \frac{1}{\lambda^2} \]


Intezita poruch:

\[ \lambda(t) = \frac{f(t)}{1 - F(t)} \]

\begin{verbatim}

lambda = 1/30
interval = 10

pexp(10, 1/30)   -- get probability

quantile = 0.05

qexp(0.05, 1/30)  -- get desired interval

Pravdepodobnost, ze k T udalostem dojde drive nez v case X


Prumerna doba: 10
Pozadovany pocet udalosti = 50

mu_ti = 10
sigma_ti^2 = 10
sigma_ti = sqrt(10)

T = suma 100 Ti

T ~ N(50 * 10, sqrt(10) * 100)
T ~ N(udalosti * mu, sigma_ti * udalosti)

pnorm(X, udalosti*mu, sigma_ti * udalosti)

Vyplotovani:

x = seq(0, 1000, by=1)
y = dnorm(x, mean=udalosti*mu, sd=sigma_ti*udalosti)
png(file="aaa.png")
plot(x, y, type="l")
dev.off()

\end{verbatim}

\subsection{Weibullovo rozdeleni}

Modelovani doby do vyskytu udalosti, umoznuje modelovat obdobi casnych poruch
a obdobi starnuti

\[ X \rightarrow W(\Theta, \beta) \]

$\Theta \rightarrow$ parametr meritka

$\beta \rightarrow$ parametr tvaru

\[ \lambda(t) = konstatnta * t^{\beta - 1} \]

Distribucni funkce = 

\begin{equation}
    F(t)=\begin{cases}
        1 - e^{-\frac{t}{\Theta}^\beta}, & t > 0\\
        0, & \text{jinak}.
  \end{cases}
\end{equation}

Hustota p.

\begin{equation}
    F(t)=\begin{cases}
        \frac{\beta}{\Theta^{\beta}} t^{\beta - 1} e^{-\frac{t}{\Theta}^\beta}, & t > 0\\
        0, & \text{jinak}.
  \end{cases}
\end{equation}

Intenzita poruch:

\[ \lambda(t) = \frac{\beta}{\Theta^\beta t^{\beta - 1}} \]

Priklad:
$\Theta = 50$

Intenzita poruch je linearni a rostouci, tedy

$\beta = 2$

Hodnotu $\beta$ ziskame z nasledujiciho vzorce

\[ \lambda(t) = konstatnta * t^{\beta - 1} \]

Casovy interval - deset

\[ X \rightarrow W(\Theta = 50, \beta = 2) \]

Intenzita poruch -- dosazenim do vzorce

\[ \lambda(10) = \frac{2}{50^2} 10^{2-1} = 0.008 \]

Pravdepodobnost, ze system bude 100 hodin bezporuchovy

\[ P(X > 100) = 1 - F(100) \]

\begin{verbatim}

pweibull(100, beta, Theta)

1 - pweibull(100, 2, 50)

\end{verbatim}

\subsection{Erlangovo rozdeleni}

Doba vyskytu do $k$-te udalosti v Poissonove procesu

$k$ -- pocet udalosti

$\lambda$ -- meritko

\[ X_k \rightarrow Erlang(k, \lambda) \]

\[ E(X_k) = \frac{k}{\lambda} \]

\[ D(X_k) = \frac{k}{\lambda^2} \]

\subsection{Normalni rozdeleni}

$\mu$ -- stredni hodnota

$\sigma^2$ -- rozptyl

\[ X \rightarrow N(\mu; \sigma^2) \]

Pravidlo 3$\sigma$

99.8 \% prvku spada do intervalu $<\mu - 3\sigma; \mu + 3\sigma>$

Q-Q graf = graficky nastroj pro overeni normality

\begin{verbatim}
qqline(data, col="blue")
\end{verbatim}

\section{Statistika}

Promenne

\begin{itemize}
    \item Kvalitativni
        \begin{itemize}
            \item Nominalni = nelze sortit
            \item Ordinalni = lze sortit
        \end{itemize}
    \item Kvantitativni
        \begin{itemize}
            \item Diskretni
            \item Spojite
        \end{itemize}
\end{itemize}

\subsection{Nominalni hodnota}

Cetnost

Relativni cetnost

\[ p_i = \frac{n_i}{n} \]

Modus -- nejcastejsi prvek

Histogram, vysecovy graf

\subsection{Ordinalni promenna}

Kumulativni cetnost, kumulativni relativni cetnost

Soucet prvku varianty $x$ nebo nizsi

Lorenzova krivka -- vynasim kumulativni cetnosti

Paretova analyza, Paretuv princip - pravidlo $\frac {20} {80}$

\subsection{Numericke promenne}

Mira polohy a variability

Prumer = $\bar{x}$

Vlastnosti:
\begin{itemize}
    \item Soucet odchylek od prumeru je roven nule
    \item Pricteme-li ke vsem hodnotam stejne cislo, o stejne cislo se zvedne prumer
    \item vynasobime-li vsechny hodnoty stejnym cislem, stejnym pomerem se zvysi prumer
\end{itemize}

\textbf{Harmonicky prumer}

Cast z celku, typicky uloha o spolecne praci

\[ \bar{x}_H = \frac {n} { \sum_{i=1}^n \frac{1}{x_i} } \]

Vazeny prumer

\[ \bar{x}_H = \frac { \sum_{i-1}^k n_i } { \sum_{i=1}^n \frac{n_i}{x_i} } \]

\textbf{Geometricky prumer}

Relativni zmena

\[ \bar{x}_G = \sqrt[n]{ x^{n_1}_1 * x^{n_2}_2 * ... * x^{n_k}_n } \]

Modus:

Pro diskretni hodnotu to je nejcetnejsi hodnota

Pro spojitou to je hodnota, okolo ktere je nejvyssi koncentrace hodnot --
urcujeme pomoci \textbf{shorthu} - co nejkratsi interval takovy, ze v nem lezi
alespon 50 \% hodnot. Modus je potom stred shorthu.

Kvantil -- rozdeluje dataset na dve casti, mensi nez kvantil a vetsi nez kvantil

\textbf{Interkvartilove rozpeti IQR}

\[ IQR = x_{0.75} - x_{0.25} \]

\textbf{MAD}

Mean Absolute Deviation

median absolutnich odchylek kazde hodnoty od medianu

\textbf{Vyberovy rozptyl}

\[ s^2 = \frac { \sum_{i=1}^{n} (x_i - \bar{x})^2 } {n-1} \]

Suma kvadratu odchylek od prumeru podeleno velikosti datasetu bez jedne

Priceteme-li ke vsem hodnotam konstantu, rozptyl se nezmeni

Vynasobime-li vsechny hodnoty konstantou, rozptyl se prenasobi kvadratem konstanty

\textbf{Vyberova smerodatna odchylka}

\[ \sqrt{s^2} \]

\textbf{Variacni koeficient}

Vyjadruje miru variability promenne $x$, lze stanovit pro promenne, ktere nabyvaji pouze kladnych hodnot pomoci vztahu

\[ V_x = /frac V {\bar{x}} \]

\subsection{Identifikace outliers}

\textbf{Vnitrni hradby}

\[ x_i < x_{0.25} - 1.5 * IQR \]

nebo

\[ x_i > x_{0.75} + 1.5 * IQR \]

Pak $x_i$ je outlier

\textbf{z-souradnice}

z-skore$_i$ = $ \frac{x_i - \bar{x}} {s} $

$|\text{z-skore$_i$}| > 3 \implies |  \left| \frac{x_i - \bar{x}}{s} \right| > 3 \implies |x_i - \bar{x}| > 3s \implies$ $x_i$ je outlier

\textbf{$x_{0.5}$-souradnice}

\[ |x_{0.5} - skore_i| = \left| \frac{x_i - x_{0.5}}{1.483 MAD} \right| > 3 \implies x_i \text{ je outlier} \]

\textbf{Odlehla a extremni pozorovani}

Odlehla:

\[ h_D = x_{0.25} - 1.5IQR \]
\[ h_D = x_{0.75} + 1.5IQR \]

Extremni: 

\[ H_D = x_{0.25} - 3IQR \]
\[ H_D = x_{0.75} + 3IQR \]

\textbf{Vyberova sikmost}

\[ a = \frac{n}{(n-1)(n-2)} * \frac{ \sum_{i=1}^{n} (x_i - \bar{x})^3 }{s^3} \]

a > 0 ... prevazuji hodnoty mensi nez prumer
a < 0 ... prevazuji hodnoty vetsi nez prumer
a = 0 ... symetricke rozlozeni

\textbf{Vyberova spicatost}

b = 0 ... odpovida normalnimu rozdeleni
b > 0 ... spicate rozdeleni
b < 0 ... ploche rozdeleni

\subsection{Graficke znazorneni}

\textbf{Box plot}

\begin{verbatim}
png(file='boxplot.png')
boxplot(machine1, machine2, machine3, machine4, 
        main='Prumery lozisek', ylab='mm', names=machine.names)
dev.off()
\end{verbatim}

\subsection{Vyberove charakteristiky}

Vyberovy prumer

\textbf{Zakon velkych cisel}

S rostoucim rozsahem vyberu se vyberovy prumer koncentruje stale vice okolo skutecneho prumeru

\textbf{Centralni limitni veta}

Nezavisle na rozdeleni, z ktereho $X_i$ pochazi, se pro dostatecne velky vyber rozdeleni prumeru
blizi normalnimu rozdeleni

Vyber neobsahuje odlehla pozorovani a rozsah je alespon 30

Viz priklad u Exp rozdeleni

\subsection{Relativni cetnost}

\[ \bar{X} = \frac{\sum_{i=1}^{n} n}{X_i} = p \]

Vlastnosti

\[ E(p) = \mu_p \]
\[ D(p) = \sigma_p^2\ = \frac{\pi (1 - \pi)}{n} \]

\[p \sim N(\mu_p, \sigma_p^2)\]


\textbf{Rozdil vyberovych prumeru}

Vyber je max. dvacetina populace

Vybery jsou nezavisle

Plati predpoklady CLV -- vybery pochazi z norm. rozdeleni, nebo jsou dostatecne velke (30+)

Pak plati:

\[ E(\bar{X}_1 - \bar{X}_2) = \mu_1 - \mu_2 \]

\[ D(\bar{X}_1 - \bar{X}_2) = \frac{\sigma_1^2}{n_1} + \frac{sigma_2^2}{n_2} \]

\textbf{Rozdil relativnich cetnosti}

Rozsah kazde z populaci je dostatecne velky (vyber je max desetina populace)

Pro modelovani rozdilu lze pouzit norm. rozdeleni (dostatecne velke vybery)

Vybery jsou nezavisle

Pak plati:

\[ E(p_1 - p_2) = \pi_1 - \pi_2 \]

\[ D(p_1 - p_2) = \frac{\pi_1(1 - \pi_1)}{n1} + \frac{\pi_2(1 - \pi_2)}{n2}\]

\[ (p_1 - p_2) \sim N(E(p_1 - p_2), D(p_1 - p_2)) \]

\subsection{$\chi^2$ rozdeleni}

Soucet ctvercu nahodnych velicin s normovanych normalnim rozdelenim

Pocet nahodnych velicin = $\nu$ stupnu volnosti

\[ X = \sum_{i=1}^{v} Z_i^2 \rightarrow \chi^2 \]

Plati:

\[ \frac{(n-1) S^2}{\sigma^2} = \chi^2_{n-1} \]

\[E(X) = \nu, D(X) = 2\nu\]

Pouziti:

Test, zda rozptyl souboru s norm. rozdelenim je roven $\sigma_0^2$

Overeni nezavislosti kategorialnich promennych

Test dobre shody - zda nahodne veliciny pochazi z urciteho rozdeleni

Priklad:

$\mu = 5 \text{let}$

$\sigma = 6 \text{mesicu}$

$P(S > 7) = ?$

\[ \frac{(n-1) S^2}{\sigma^2} \rightarrow \chi^2_{n-1} \]

$X = \frac{(n-1)S^2}{\sigma^2}$

\[ X \rightarrow \chi_{19}^2 \]

$(X > \frac{19.7^2}{36}), \text{ tedy } (X > 25.86)$

\[ 1 - F_{\chi_{19}^2} = 0.134 \]

\begin{verbatim}
1 - pchisq(q, df)
1 - pchisq(25.86, 19)
\end{verbatim}

\subsection{Studentovo rozdeleni}

Z - nahodna velicina o norm. rozdeleni
V - nahodna velicina o $\chi^2$ rozdeleni s $\nu$ stupni volnosti

\[ T = \frac{Z}{\sqrt{\frac{V} {\nu}}} \]

T ma Studentovo rozdeleni s $\nu$ stupni volnosti

Pro $\nu$ vyssi nez 30 se Studentovo rozdeleni blizi normovanemu norm. rozdeleni

\[E(T) = 0 \text{ pro } \nu > 0\]

\[D(T) = \frac{\nu}{\nu - 2} \text{pro} \nu > 2 \]

\textbf{Vlastnosti}

Pokud $X_1, X_2, ..., X_3$ maji norm. rozdeleni a jsou navzajem nezavisle, potom

\[ \frac{\bar{X} - \mu} {S} \sqrt{n} \]

ma Studentovo rozdeleni $t$ s $n-1$ stupni volnosti

\[ \frac{\bar{X} - \mu} {S} \sqrt{n} \rightarrow t_{n-1} \]

Pouziti:

Testovani hypotez o stredni hodnote, pokud je rozptyl neznamy
a vyber pochazi z norm. rozdeleni

Testovani hypotez o shode strednich hodnot, pokud mame dva nezavisle vybery, jejichz
rozptyly jsou nezname, ale shodne

Analyza vysledku regresni analyzy

\subsection{Fisherovo-Snedecorovo rozdeleni}

Dve nezavisle veliciny, V a W s rozdelenim $\chi^2$. Prvni ma $m$ stupnu volnosti,
druhe $n$ stupnu volnosti. Pak nahodna velicina

\[ F = \frac {\frac{V}{m}} {\frac{W}{n}} \]

ma FS rozdeleni o $m$ a $n$ stupnich volnosti

\[ F \rightarrow F_{m,n} \]

\[ E(F) = \frac n {n-2} \text{pro} n>2 \]
\[ D(F) = \frac {2n^2 (1 + \frac {n-2}{m})} {(n-2)^2 (n-4)} \text{pro} n>4 \]

Pouziti:

Test shody rozptylu dvou souboru

Test shody strednich hodnot vice nez dvou zakladnich souboru

Testy v regresni analyze

\subsection{Teorie odhadu}

\textbf{Bodove odhahy}

\textbf{Intervalove odhady}

Spolehlivost: $1 - \alpha$

Hladina vyznamnosti: $\alpha$

Odhad stredni hodnoty, znam-li smerodatnou odchylku

\[ \bar{X} \sim N\left( \mu; \frac{\sigma^2}{n} \right) \]

\subsection{Testovani hypotez}

Parametricke testy

Neparametricke testy

Nulova a alternativni hypoteza

Zamitame nebo nezamitame hypotezu -- nelze prijmout nebo potvrdit

Chyba I. druhu -- nespravne zamitneme
Chyba II. druhu -- nespravne nezamitneme

Dva pristupy:

Klasicky test

Cisty test vyznamnosti

\subsection{Klasicky test}

formulace hypotez

Volba statistiky

Stanoveni $\alpha$

Sestrojeni kritickeho oboru

Vypocet pozorovane hodnoty

Formulace zaveru

\subsection{Cisty test vyznamnosti}

Formulace hypotez

Volba test. statistiky

Vypocet pozorovane hodnoty

Vypocet p-hodnoty

Rozhodnuti na zaklade p-hodnoty

\subsection{Test o rozptylu}

$n=10$

$H_0: \sigma^2 = \sigma^2_0$

Priklad

$H_0: \sigma^2 = 4$

$\bar{x} = 170.3 \text{g}$

$s^2$ - suma kvadratu odchylek deleno n-1

$s^2 = 5,3 g^2$

Normalita ze zadani

\subsection{Jednovyberovy t-test}

\end{document}


