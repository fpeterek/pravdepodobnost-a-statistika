\documentclass{article}
\usepackage[utf8]{inputenc}
\usepackage{xurl}
\usepackage{amsmath}
\usepackage{bm}
\usepackage[czech]{babel}

\setlength{\parskip}{\baselineskip}
\setlength{\parindent}{0pt}

\newcommand{\fakesection}[1]{%
  \par\refstepcounter{section}% Increase section counter
  \sectionmark{#1}% Add section mark (header)
  \addcontentsline{toc}{section}{\protect\numberline{\thesection}#1}% Add section to ToC
  % Add more content here, if needed.
}

\title{Pravděpodobnost a statistika}
\date{15. květen 2022}
\author{Filip Peterek}

\begin{document}

\maketitle

\section{Pravděpodobnost}

\subsection{Zakladni vzorce}

Variace bez opakovani: \[V(n,k) = \frac{n!}{(n-k)!}\]

Kombinace bez opakovani: \[C(n,k) = \frac{n!}{k!(n-k)!}\]

Permutace: \[P(n) = n!\]

Variace s opakovanim: \[V*(n,k) = n^k\]

Kombinace s opakovanim: \[C*(n,k) = C*(n+k-1, k) = \frac{(n+k-1)!}{(n-1)! * k!}\]

Permutace s opakovanim: \[P*(n_1, n_2, ..., n_k) = \frac{P(n)}{P(n_1) * P(n_2) * ... * P(n_k)} = \frac{n!}{n_1! * n_2! * ... * n_3!} \]

Prunik jevu: \[P(A \cap B) = P(A|B) * P(B)\]

Prunik nezavislych jevu: \[P(A \cap B) = P(A) * P(B)\]

Podminena pravdepodobnost: \[P(A|B) = \frac{P(A \cap B)}{P(B)}\]

\subsection {Bayesuv vzorec}

Nastal jev A, hledam pravdepodobnost, ktery z jevu $B_i$ jev A zpusobil.

\[ P(B_k|A) = \frac{P(A|B_k) * P(B_k)}{ \sum_{i=1}^{n} P(A|B_i) * P(B_i) } \]
 
\subsection{Nahodna velicina}

Stredni hodnota: 
\[ \mu = \sum_{(i)} x_i * P(x_i) \]
\[ \mu = \int_{-\infty}^{\infty} x_i * P(x_i) \]

\[ E(aX + b) = aE(X) + b \]
\[ E(\sum_i^n X_i) = \sum_i^n E(X_i) \]

Centralni moment r-teho radu:

\[ \mu_r' = \sum_{(i)} (x_i - E(X))^r * P(x_i) \]
\[ \mu_r' = \int_{-\infty}^{\infty} (x_i - E(X))^r * P(x_i) \]

Variance:

\[ D(X) = \sum_{(i)} (x_i - E(X))^2 * P(x_i) \]

\[ D(X) = \int_{-\infty}^{\infty} (x_i - E(X))^2 * P(x_i) \]

\[ D(X) = E(X^2) - (E(X))^2 \]

\[ D(aX + b) = a^2D(X) \]

Smerodatna odchylka:

\[ \sigma = \sqrt{D(X)} \]

Sikmost:

\[ \alpha_3 = \frac {\mu_3} {\sigma^3} \]

Spicatost:

\[ \alpha_4 = \frac {\mu_4} {\sigma^4} \]

Modus: nejpocetnejsi prvek, prvek s nejvyssi pravdepodobnosti

\subsection{Nahodny vektor}

Vektor, jehoz slozky jsou nahodne veliciny

Vztahy jsou ekvivalentni nahodne velicine, ale upravene pro vektor

Ukazka:

Necht $\boldsymbol{X} = (X, Y)$ je nahodny vektor. Potom plati:

\[ E(\boldsymbol{X}) = (E(X), E(Y)) \]

\subsection{Nezavislost nahodnych velicin}

Necht $\boldsymbol{X} = (X, Y)$ je nahodny vektor. X, Y jsou nezavisle, prave kdyz plati:

\[ F(x,y) = F_X(x) * F_Y(y) \]

\subsection{Kovariance a koeficient korelace}

\textbf{Kovariance $cov(X,Y)$}

\[ cov(X,Y) = E[(X - E(X)) * (Y - E(Y))] \]

Kladna hodnota kovariance: zvysi se X $\implies$ pravdepodobne se zvysi Y
Zaporna hodnota kovariance: zvysi se X $\implies$ pravdepodobne se snizi Y

\[ cov(X, Y) = E(XY) - E(X)*E(Y) \]
\[ cov(X,X) = D(X) \]
\[ cov(a_1X + b_1, a_2X + b_2) = a_1a_2cov(X,Y) \]

Jsou-li X, Y nezavisle $\implies$ $cov(X,Y) = 0$

\textbf{Korelacni koeficient $\rho(X, Y)$}

\begin{equation}
    \rho(X,Y)=\begin{cases}
    \frac { cov(X,Y) } { \sqrt{D(X) * D(Y)} }, & D(X), D(Y) \neq 0\\
    0, & \text{jinak}.
  \end{cases}
\end{equation}

Korelacni koeficient je mirou linearni zavislosti dvou slozek nahodneho vektoru.

\[ \rho(X, Y) = \rho(Y, X) \]

\[ \rho(X, X) = 1 \]

\[ X, Y \text{ jsou nezavisle } \implies \rho(X, Y) = 0 \]

Implikace, naopak predchozi vztah neplati

\[ \rho(X, Y) = 0 \implies X, Y jsou nekorelovane \]

\subsection{Alternativni rozdeleni}

Pouze dve moznosti, kazde ma svou pravdepodobnost

\[ P(X=1) = p \]
\[ P(X=0) = 1-p \]

\[ E(X) = p, D(X) = p * (1 - p) \]

\subsection{Binomicke rozdeleni}

Provadim nezavisle pokusy (Bernoulliho pokusy), pravdepodobnost uspechu je konstantni

Binomicke rozdeleni - pravdepodobnost, ze v $x$ pokusech se objevi $y$ uspechu

\begin{verbatim}

20 pokusu
Pravdepodobnost jednoho uspechu je 0.3
Pravdepodobnost, ze uspechu bude pet a mene ziskame

pbinom(5, 20, 0.3)

Pravdepodobnost, ze uspechu bude nad pet

1 - pbinom(5, 20, 0.3)


Pocet uspechu je 6
Pozadovana pravdepodobnost pro 6 uspechu je 0.7
Pravdepodobnost uspechu pri jednom pokusu je 0.3

qnbinom(0.7, 6, 0.3) + 6

Je treba pricist 6, bo R pocita jen neuspechy, 
kdezto my chceme vsechny pokusy

nbinom - negativne binomicke rozdeleni

\end{verbatim}

\end{document}

